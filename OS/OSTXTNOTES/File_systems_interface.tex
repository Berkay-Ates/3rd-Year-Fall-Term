                            File Systems 
=> We just think and use files when we want to store something 
=> OS and its FILE SYSTEM components provide users special tools to write 
   files into the memory 

=> The content of the file have to save int Hard Disk 


FILE CONCEPT 
-> Contiguous logical address space 
-> Data 
    - numeric 
    - character
    - binary 
-> Program 

                            File Structure 
=> None - sequence of word bytes 
=> Simple record structure 
    - Lines 
    - Fixed length 
    - Variable length 
=> Complex Structures 
    - Formatted documents 
    - Relocatable load file 
=> Can simulate last two with method by insertingn appropriate control characters 
=> Who decides:
    - OS 
    - Program 


                        FILE ATTRIBUTES 
NAMES: Only information kept in human readable form 
Identifier: Unique tag (number) identifies file within file system 
Type: Needed for systems that support different types 
Location: Pointer to file location on device 
Size: Current file size 
Protection: Controls who can do reading, writing, executing
Time,data,and user identification: Data for protection, security and usage 
monitoring 

=> Information about files are kept in the directory structure which is
maintained on the disk


FILES AND DIRECTORIES 
-> There are two bacis things are stored on disk as part of the area controlled 
   by the fire system 

   - files (store ocntent)
   - directory information(can be a tree): keeps info about files, their
   attributes or locations 


FILE OPERATIONS 
-> File is an abstract data type 
-> Common Operations that are supported by the Operationg Systems 
    - Create 
    - Write 
    - Read 
    - Reposition within file 
    - Delete 
    - Truncate 

-> Open(F) - search the directory structure on disk for entry F, and move the 
    content of entry to meomory 

-> Close(f) - move the content of entry F, in memory to directory structure 
   on disk 


OPEN FILES 
- When a file is not opened no information about that file would be kept in 
memory. However following informations kept in memory when a file opened 

Several pieces of data are needed to manage open files: 
-File pointer: pointer to last read/write location, per process that has 
               the file open 
-File open count: counter of number of tiems a file is open- to allow 
                removal of data from open-file table wehen last processes
                doeses it 
-Disk location of the file: cache of data access information 
-Access rights: per-process access mode information 


File Locking 
-Provided by some operating systems and file systems 
-Mediates access to a file 
-Mandatory or advisory: 
    - Mandatory-> access is denied depending on locks held and requested
    - Advisory-> processes can find status of locks and decide what to do 


                            Access Methods 
-> Sequential Access 
    - read next 
    - write next 
    - reset 
    - no read after last write 

-> Direct Access 
    - read n 
    - write n 
    - position to n read next, write next 
    - rewrite n 
n=relative block number 

                            File Sharing 
-Each user that has an account in the computer has a username and a unique 
user ID (UID)
-The administrator can create groups. A group may have a set of usernames 
(users) associated with it. Each group has a unique group ID (GID)


                    FILE SHARING - MULTIPLE USERS 
                            PROTECTION 
- Protection is based on the use of UserIDs and GroupIDs 
- Each file has associated protection bits (permissions) for userID and groupID 
        - User ID: read, write, execute? 
        - Group ID: read, write, execute?
        
user IDs: IDentify users, allowing permissions and protections to be per-user 
Group IDs: allow users to be in groups, permitting group access rights 


                    FILE SHARING - CONSISTENCY SEMANTICS 
Consistency semantics: specify how multiple users are to access a shared 
                       file simultaneously 


    -> Unix file system UFS implementation
        -Writes to an open file visible immediately to other users of the 
        same open file 

    -> AFS has session semantics 
        -> Writes only visible to sessions starting after the file is closed 



                        PROTECTION
-> File owner/creator should be able to control
    - what can be done (read,write,execute...)
    - by whom (owner,others,group member )

-> Types of access (what can be done)
-Read 
-Write
-Execute 
-Append
-Delete
-List 

                        ACCESS LIST AND GROUPS     
-> Mode of access: read, write, execute 
-> Three classes of users 
                            RWX
        a- owner access     111
        b-group access      110
        c-public access     001

-> Ask manager to create a group (unique name), say G, and add some users 
to the group 
-> For a particular file (say game) or subdirectory define an appropriate access 

                            chmod test game 
Attach a group to a file 
                            chgrp G game 






































